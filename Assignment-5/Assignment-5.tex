\documentclass{assignment}
\ProjectInfos{量子光学}{PHYS6651P}{2021-2022 学年第二学期}{作业五}{截止时间: 2022 年 5 月 20 日 (周五)}{陈稼霖}[https://github.com/Chen-Jialin]{SA21038052}

\begin{document}
\begin{prob}
    \[
        \dot{\rho}=-\frac{\mathscr{C}}{2}\bar{n}_{\text{th}}(aa^{\dagger}\rho-2a^{\dagger}\rho a+\rho aa^{\dagger})-\frac{\mathscr{C}}{2}(\bar{n}_{\text{th}}+1)(a^{\dagger}a\rho-2a\rho a^{\dagger}+\rho a^{\dagger}a)
    \]
    单模场同热库耦合, 求 $\langle a^{\dagger}a\rangle(t)$, 即 $\langle\hat{n}(t)\rangle$, 初始为 $n(0)$.
\end{prob}
\begin{sol}
    $\langle a^{\dagger}a\rangle$ 的运动方程为
    \begin{align}
        \notag&\frac{\mathrm{d}\langle a^{\dagger}a\rangle}{\mathrm{d}t}=\frac{\mathrm{d}}{\mathrm{d}t}\tr(\rho a^{\dagger}a)=\tr(\dot{\rho}a^{\dagger}a)\\
        \notag=&\tr\left\{\left[-\frac{\mathscr{C}}{2}\bar{n}_{\text{th}}(aa^{\dagger}\rho-2a^{\dagger}\rho a+\rho aa^{\dagger})-\frac{\mathscr{C}}{2}(\bar{n}_{\text{th}}+1)(a^{\dagger}a\rho-2a\rho a^{\dagger}+\rho a^{\dagger}a)\right]a^{\dagger}a\right\}\\
        \notag=&\tr\left\{-\frac{\mathscr{C}}{2}\bar{n}_{\text{th}}(aa^{\dagger}\rho a^{\dagger}a-2a^{\dagger}\rho aa^{\dagger}a+\rho aa^{\dagger}a^{\dagger}a)-\frac{\mathscr{C}}{2}(\bar{n}_{\text{th}}+1)(a^{\dagger}a\rho a^{\dagger}a-2a\rho a^{\dagger}a^{\dagger}a+\rho a^{\dagger}aa^{\dagger}a)\right\}\\
        \notag=&\tr\left\{\rho\left[-\frac{\mathscr{C}}{2}\bar{n}_{\text{th}}(a^{\dagger}aaa^{\dagger}-2aa^{\dagger}aa^{\dagger}+aa^{\dagger}a^{\dagger}a)-\frac{\mathscr{C}}{2}(\bar{n}_{\text{th}}+1)(a^{\dagger}aa^{\dagger}a-2a^{\dagger}a^{\dagger}aa+a^{\dagger}aa^{\dagger}a)\right]\right\}\\
        \notag=&\tr\left\{\rho\left[-\frac{\mathscr{C}}{2}\bar{n}_{\text{th}}(a^{\dagger}a(a^{\dagger}a+1)-2(a^{\dagger}a+1)(a^{\dagger}a+1)+(a^{\dagger}a+1)a^{\dagger}a)-\frac{\mathscr{C}}{2}(\bar{n}_{\text{th}}+1)(2a^{\dagger}(a^{\dagger}a+1)a-2a^{\dagger}a^{\dagger}aa)\right]\right\}\\
        \notag=&\tr\left\{\rho\left[-\frac{\mathscr{C}}{2}\bar{n}_{\text{th}}(-2a^{\dagger}a-2)-\frac{\mathscr{C}}{2}(\bar{n}_{\text{th}}+1)(2a^{\dagger}a)\right]\right\}\\
        \notag=&\tr\{\mathscr{C}\rho(\bar{n}_{\text{th}}-a^{\dagger}a)\}\\
        =&\mathscr{C}(n_{\text{th}}-\langle a^{\dagger}a\rangle).
    \end{align}
    考虑到 $\langle a^{\dagger}a\rangle$ 的初始值为 $n(0)$, 解上式得
    \begin{align}
        \langle a^{\dagger}a\rangle(t)=[n(0)-\bar{n}_{\text{th}}]e^{-\mathscr{C}t}+\bar{n}_{\text{th}}.
    \end{align}
\end{sol}

\begin{prob}
    某系统与环境相互作用 Hamiltonian 为:
    \[
        \hat{V}=\Delta\sigma_z\sum_{\bm{k}}(a_{\bm{k}}^{\dagger}e^{-i\omega t}+a_{\bm{k}}e^{i\omega t})
    \]
    环境算符 $(a_{\bm{k}}^{\dagger},a_{\bm{k}})$, 环境为热平衡谐振子热库, 求系统演化的主方程 $\dot{\rho}_{s}$ (依据课本 (8.1.7) 式).
\end{prob}
\begin{sol}
    
\end{sol}
\end{document}