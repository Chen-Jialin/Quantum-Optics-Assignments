\documentclass{assignment}
\ProjectInfos{量子光学}{PHYS6651P}{2021-2022 学年第二学期}{作业五}{截止时间: 2022 年 5 月 20 日 (周五)}{陈稼霖}[https://github.com/Chen-Jialin]{SA21038052}

\begin{document}
\begin{prob}
    \[
        \dot{\rho}=-\frac{\mathscr{C}}{2}\bar{n}_{\text{th}}(aa^{\dagger}\rho-2a^{\dagger}\rho a+\rho aa^{\dagger})-\frac{\mathscr{C}}{2}(\bar{n}_{\text{th}}+1)(a^{\dagger}a\rho-2a\rho a^{\dagger}+\rho a^{\dagger}a)
    \]
    单模场同热库耦合, 求 $\langle a^{\dagger}a\rangle(t)$, 即 $\langle\hat{n}(t)\rangle$, 初始为 $n(0)$.
\end{prob}
\begin{sol}
    $\langle a^{\dagger}a\rangle$ 的运动方程为
    \begin{align}
        \notag&\frac{\mathrm{d}\langle a^{\dagger}a\rangle}{\mathrm{d}t}=\frac{\mathrm{d}}{\mathrm{d}t}\tr(\rho a^{\dagger}a)=\tr(\dot{\rho}a^{\dagger}a)\\
        \notag=&\tr\left\{\left[-\frac{\mathscr{C}}{2}\bar{n}_{\text{th}}(aa^{\dagger}\rho-2a^{\dagger}\rho a+\rho aa^{\dagger})-\frac{\mathscr{C}}{2}(\bar{n}_{\text{th}}+1)(a^{\dagger}a\rho-2a\rho a^{\dagger}+\rho a^{\dagger}a)\right]a^{\dagger}a\right\}\\
        \notag=&\tr\left\{-\frac{\mathscr{C}}{2}\bar{n}_{\text{th}}(aa^{\dagger}\rho a^{\dagger}a-2a^{\dagger}\rho aa^{\dagger}a+\rho aa^{\dagger}a^{\dagger}a)-\frac{\mathscr{C}}{2}(\bar{n}_{\text{th}}+1)(a^{\dagger}a\rho a^{\dagger}a-2a\rho a^{\dagger}a^{\dagger}a+\rho a^{\dagger}aa^{\dagger}a)\right\}\\
        \notag=&\tr\left\{\rho\left[-\frac{\mathscr{C}}{2}\bar{n}_{\text{th}}(a^{\dagger}aaa^{\dagger}-2aa^{\dagger}aa^{\dagger}+aa^{\dagger}a^{\dagger}a)-\frac{\mathscr{C}}{2}(\bar{n}_{\text{th}}+1)(a^{\dagger}aa^{\dagger}a-2a^{\dagger}a^{\dagger}aa+a^{\dagger}aa^{\dagger}a)\right]\right\}\\
        \notag=&\tr\left\{\rho\left[-\frac{\mathscr{C}}{2}\bar{n}_{\text{th}}(a^{\dagger}a(a^{\dagger}a+1)-2(a^{\dagger}a+1)(a^{\dagger}a+1)+(a^{\dagger}a+1)a^{\dagger}a)-\frac{\mathscr{C}}{2}(\bar{n}_{\text{th}}+1)(2a^{\dagger}(a^{\dagger}a+1)a-2a^{\dagger}a^{\dagger}aa)\right]\right\}\\
        \notag=&\tr\left\{\rho\left[-\frac{\mathscr{C}}{2}\bar{n}_{\text{th}}(-2a^{\dagger}a-2)-\frac{\mathscr{C}}{2}(\bar{n}_{\text{th}}+1)(2a^{\dagger}a)\right]\right\}\\
        \notag=&\tr\{\mathscr{C}\rho(\bar{n}_{\text{th}}-a^{\dagger}a)\}\\
        =&\mathscr{C}(n_{\text{th}}-\langle a^{\dagger}a\rangle).
    \end{align}
    考虑到 $\langle a^{\dagger}a\rangle$ 的初始值为 $n(0)$, 解上式得
    \begin{align}
        \langle a^{\dagger}a\rangle(t)=[n(0)-\bar{n}_{\text{th}}]e^{-\mathscr{C}t}+\bar{n}_{\text{th}}.
    \end{align}
\end{sol}

\begin{prob}
    某系统与环境相互作用 Hamiltonian 为:
    \[
        \hat{V}=\Delta\sigma_z\sum_{\bm{k}}(a_{\bm{k}}^{\dagger}e^{-i\omega t}+a_{\bm{k}}e^{i\omega t})
    \]
    环境算符 $(a_{\bm{k}}^{\dagger},a_{\bm{k}})$, 环境为热平衡谐振子热库, 求系统演化的主方程 $\dot{\rho}_s$ (依据课本 (8.1.7) 式).
\end{prob}
\begin{sol}
    将 $\rho_S$ 的刘维尔方程截断至关于 $\hat{V}$ 的二阶项并对热库做偏迹得到课本 (8.1.7) 式:
    \begin{align}
        \dot{\rho}_S=-\frac{i}{\hbar}\tr_R[\hat{V}(t),\rho_S(t_i)\otimes\rho_R(t_i)]-\frac{1}{\hbar^2}\tr_R\int_{t_i}^t[\hat{V}(t),[\hat{V}(t'),\rho_S(t')\otimes\rho_R(t')]]\,\mathrm{d}t',
    \end{align}
    其中热平衡谐振子热库的密度算符
    \begin{align}
        \rho_R=\prod_{\bm{k}}\left[1-\exp\left(-\frac{\hbar\nu_k}{k_BT}\right)\right]\exp\left(-\frac{\hbar\nu_ka_{\bm{k}}^{\dagger}a_{\bm{k}}}{k_BT}\right),
    \end{align}
    其中 $k_B$ 为玻尔兹曼常数, $T$ 为温度.
    将相互作用 Hamiltonian 的具体表达式代入课本 (8.1.7) 式得
    \begin{align}
        \notag\dot{\rho}_S=&-\frac{i\Delta}{\hbar}\sum_{\bm{k}}\langle a_{\bm{k}}^{\dagger}\rangle[\sigma_z,\rho_S(t)]e^{-i\omega t}\\
        \notag&-\frac{\Delta^2}{\hbar^2}\int_{t_i}^t\mathrm{d}t'\,\sum_{\bm{k},\bm{k}'}\{[\sigma_z\sigma_z\rho_S(t')-2\sigma_z\rho_S(t')\sigma_z+\rho_S(t')\sigma_z\sigma_z]e^{-i\omega(t+t')}\langle a_{\bm{k}}^{\dagger}a_{\bm{k}'}^{\dagger}\rangle\\
        \notag&+[\sigma_z\sigma_z\rho_S(t')-\sigma_z\rho_S(t')\sigma_z]e^{-i\omega(t-t')}\langle a_{\bm{k}}^{\dagger}a_{\bm{k}'}\rangle\\
        \notag&+[\sigma_z\sigma_z\rho_S(t')-\sigma_z\rho_S(t')\sigma_z]e^{i\omega(t-t')}\langle a_{\bm{k}}a_{\bm{k}'}^{\dagger}\rangle\}+\text{H.C.}\\
        \notag=&-\frac{i\Delta}{\hbar}\sum_{\bm{k}}\langle a_{\bm{k}}^{\dagger}\rangle[\sigma_z,\rho_S(t)]e^{-i\omega t}\\
        \notag&-\frac{\Delta^2}{\hbar^2}\int_{t_i}^t\mathrm{d}t'\,\sum_{\bm{k},\bm{k}'}\{[\rho_S(t')-2\sigma_z\rho_S(t')\sigma_z+\rho_S(t')]e^{-i\omega(t+t')}\langle a_{\bm{k}}^{\dagger}a_{\bm{k}'}^{\dagger}\rangle\\
        \notag&+[\rho_S(t')-\sigma_z\rho_S(t')\sigma_z]e^{-i\omega(t-t')}\langle a_{\bm{k}}^{\dagger}a_{\bm{k}'}\rangle\\
        &+[\rho_S(t')-\sigma_z\rho_S(t')\sigma_z]e^{i\omega(t-t')}\langle a_{\bm{k}}a_{\bm{k}'}^{\dagger}\rangle\}+\text{H.C.}
    \end{align}
    其中
    \begin{align}
        \langle a_{\bm{k}}\rangle=\tr_R(\rho_Ra_{\bm{k}})=\langle a_{\bm{k}}^{\dagger}\rangle=\tr_R(\rho_Ra_{\bm{k}}^{\dagger})=&0,\\
        \langle a_{\bm{k}}^{\dagger}a_{\bm{k}'}\rangle=\tr_R(\rho_Ra_{\bm{k}}^{\dagger}a_{\bm{k}'})=&\bar{n}_{\bm{k}}\delta_{\bm{kk}'},\\
        \langle a_{\bm{k}}a_{\bm{k}'}^{\dagger}\rangle=\tr_R(\rho_Ra_{\bm{k}}a_{\bm{k}'}^{\dagger})=&(\bar{n}_{\bm{k}}+1)\delta_{\bm{kk}'},\\
        \langle a_{\bm{k}}a_{\bm{k}'}\rangle=\tr_R(\rho_Ra_{\bm{k}}a_{\bm{k}'})=\langle a_{\bm{k}}^{\dagger}a_{\bm{k}'}^{\dagger}\rangle=\tr_R(\rho_Ra_{\bm{k}}^{\dagger}a_{\bm{k}'}^{\dagger})=&0,
    \end{align}
    模式 $\bm{k}$ 的平均光子数
    \begin{align}
        \bar{n}_{\bm{k}}=\frac{1}{\exp\left(\frac{\hbar\nu_k}{k_BT}\right)-1},
    \end{align}
    故
    \begin{align}
        \dot{\rho}_S=-\frac{\Delta^2}{\hbar^2}\int_{t_i}^t\mathrm{d}t'\,\sum_{\bm{k}}\{[\rho_S(t')-\sigma_z\rho_S(t')\sigma_z]e^{-i\omega(t-t')}\bar{n}_{\bm{k}}+[\rho_S(t')-\sigma_z\rho_S(t')\sigma_z]e^{i\omega(t-t')}(\bar{n}_{\bm{k}}+1)\}+\text{H.C.}.
    \end{align}
    将对 $\bm{k}$ 的求和换成积分, 即
    \begin{align}
        \sum_{\bm{k}}\rightarrow 2\frac{V}{(2\pi)^3}\int_0^{2\pi}\mathrm{d}\phi\int_0^{\pi}\mathrm{d}\theta\,\sin\theta\int_0^{\infty}\mathrm{d}k\,k^2=2\frac{V}{(2\pi)^3c^3}\int_0^{2\pi}\mathrm{d}\phi\int_0^{\pi}\mathrm{d}\theta\,\sin\theta\int_0^{\infty}\mathrm{d}\nu_k\,\nu_k^2,
    \end{align}
    可得
    \begin{align}
        \notag\dot{\rho}_S=&-\frac{\Delta^2}{\hbar^2}\frac{V}{\pi^2c^3}\int_{t_i}^t\mathrm{d}t'\,\int_0^{+\infty}\mathrm{d}\nu_k\,\nu_k^2\{[\rho_S(t')-\sigma_z\rho_S(t')\sigma_z]e^{-i\omega(t-t')}\bar{n}_{\bm{k}}+[\rho_S(t')-\sigma_z\rho_S(t')\sigma_z]e^{i\omega(t-t')}(\bar{n}_{\bm{k}}+1)\}\\
        &+\text{H.C.}.
    \end{align}
    在马尔科夫近似下, $\rho_S(t')=\rho_S(t)$, 利用
    \begin{align}
        \int_{t_i}^t\mathrm{d}t'\,e^{-i\omega(t-t')}=\pi\delta(\omega),
    \end{align}
    可得
    \begin{align}
        \dot{\rho}_S=-\frac{\Delta^2}{\hbar^2}\frac{V}{\pi^2c^3}\pi\delta(\omega)\int_0^{+\infty}\mathrm{d}\nu_k\,\nu_k^2\{[\rho_S(t')-\sigma_z\rho_S(t')\sigma_z]\bar{n}_{\bm{k}}+[\rho_S(t')-\sigma_z\rho_S(t')\sigma_z](\bar{n}_{\bm{k}}+1)\}+\text{H.C.}.
    \end{align}
    % 利用
    % \begin{align}
    %     \notag\int_{-\infty}^{+\infty}\mathrm{d}\nu_k\,\nu_k^2\bar{n}_{\bm{k}}=&\int_0^{+\infty}\mathrm{d}\nu_k\,\frac{\nu_k^2}{\exp\left(\frac{\hbar\nu_k}{k_BT}+1\right)}\\
    %     \notag&(\text{令}x=\frac{\hbar\nu_k}{k_BT})\\
    %     \notag=&\left(\frac{k_BT}{\hbar}\right)^3\int_0^{+\infty}\mathrm{d}x\,\frac{x^2}{\exp(x)+1}\\
    %     \approx&2.404\left(\frac{k_BT}{\hbar}\right)^3,
    % \end{align}
    % 可得
    % \begin{align}
    %     \dot{\rho}_S=
    % \end{align}
\end{sol}
\end{document}