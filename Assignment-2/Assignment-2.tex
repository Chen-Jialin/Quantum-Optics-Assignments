\documentclass{assignment}
\ProjectInfos{量子光学}{PHYS6651P}{2021-2022 学年第二学期}{作业二}{截止时间: 2022 年 3 月 18 日 (周五)}{陈稼霖}[https://github.com/Chen-Jialin]{SA21038052}

\begin{document}
\begin{prob}
    求单模热光场的平均光子数.
\end{prob}
\begin{sol}
    单模热光场的平均光子数为
    \begin{align}
        \notag\langle N\rangle=&\tr(\rho a^{\dagger}a)\\
        \notag=&\sum_m\langle m\rvert\left[1-\exp\left(-\frac{\hbar\omega}{kT}\right)\right]\sum_n\exp\left(-n\frac{\hbar\omega}{kT}\right)\lvert n\rangle\langle n\rvert a^{\dagger}a\lvert m\rangle\\
        \notag=&\left[1-\exp\left(-\frac{\hbar\omega}{kT}\right)\right]\sum_{m,n}\exp\left(-n\frac{\hbar\omega}{kT}\right)\langle m\vert n\rangle\langle n\rvert a^{\dagger}a\lvert m\rangle\\
        \notag=&\left[1-\exp\left(-\frac{\hbar\omega}{kT}\right)\right]\sum_n\exp\left(-n\frac{\hbar\omega}{kT}\right)\langle n\rvert a^{\dagger}a\lvert n\rangle\\
        \notag=&\left[1-\exp\left(-\frac{\hbar\omega}{kT}\right)\right]\sum_n\exp\left(-n\frac{\hbar\omega}{kT}\right)n\\
        \notag=&\frac{\exp\left(-\frac{\hbar\omega}{kT}\right)}{1-\exp\left(-\frac{\hbar\omega}{kT}\right)}\\
        =&\frac{1}{\exp\left(\frac{\hbar\omega}{kT}\right)-1}.
    \end{align}
\end{sol}

\begin{prob}
    证明平移算符的性质:
    \[
        D^{\dagger}=D(-\alpha)=[D(\alpha)]^{-1},
    \]
    \[
        D^{-1}(\alpha)aD(\alpha)=a+\alpha.
    \]
\end{prob}
\begin{pf}
    位移算符
    \begin{align}
        D(\alpha)=e^{\alpha a^{\dagger}-\alpha^*a}=e^{\alpha a^{\dagger}}e^{-\alpha^*a}e^{-\abs{\alpha}^2/2}.
    \end{align}

    先证第一个等式: 由于
    \begin{align}
        D^{\dagger}(\alpha)=&e^{\alpha^*a-\alpha a^{\dagger}},\\
        D(-\alpha)=&e^{-\alpha a^{\dagger}+\alpha^*a},\\
        [D(\alpha)]^{-1}=&e^{-\alpha a^{\dagger}+\alpha^*a},
    \end{align}
    故
    \begin{align}
        D^{\dagger}(\alpha)=D(-\alpha)=[D(\alpha)]^{-1}.
    \end{align}

    再证第二个等式: 由于 $[a,[a,a^{\dagger}]]=[a^{\dagger},[a,a^{\dagger}]]=0$, 利用 Baker-Hansdoff 公式, 有
    \begin{align}
        D^{-1}(\alpha)=e^{\alpha^*a-\alpha a^{\dagger}}=e^{\alpha^*a}e^{-\alpha a^{\dagger}}e^{-[\alpha^*a,\alpha a^{\dagger}]/2}=e^{\alpha^*a}e^{-\alpha a^{\dagger}}e^{\abs{\alpha}^2/2},
    \end{align}
    从而
    \begin{align}
        D^{-1}(\alpha)aD(\alpha)=e^{\alpha^*a}e^{-\alpha a^{\dagger}}e^{\abs{\alpha}^2/2}ae^{\alpha a^{\dagger}}e^{-\alpha^*a}e^{-\abs{\alpha}^2/2}=e^{\alpha^*a}e^{-\alpha a^{\dagger}}ae^{\alpha a^{\dagger}}e^{-\alpha^*a},
    \end{align}
    利用 $e^{-xa^{\dagger}}ae^{xa^{\dagger}}=a+x$ 得
    \begin{align}
        D^{-1}(\alpha)aD(\alpha)=e^{\alpha^*a}(a+x)e^{-\alpha^*a}=a+\alpha.
    \end{align}
\end{pf}

\begin{prob}
    证明:
    \[
        a^{\dagger}\lvert\alpha\rangle\langle\alpha\rvert=(\alpha^*+\frac{\partial}{\partial\alpha})\lvert\alpha\rangle\langle\alpha\rvert,
    \]
    \[
        \lvert\alpha\rangle\langle\alpha\rvert a=(\alpha+\frac{\partial}{\partial\alpha^*})\lvert\alpha\rangle\langle\alpha\rvert.
    \]
\end{prob}
\begin{pf}
    相干态
    \begin{align}
        \lvert\alpha\rangle=e^{-\abs{\alpha}^2/2}\sum_n\frac{\alpha^n}{\sqrt{n!}}\lvert n\rangle,
    \end{align}
    故
    \begin{align}
        \lvert\alpha\rangle\langle\alpha\rvert=e^{-\abs{\alpha}^2}\sum_{n,m}\frac{\alpha^n}{\sqrt{n!}}\frac{(\alpha^*)^m}{\sqrt{m!}}\lvert n\rangle\langle m\rvert.
    \end{align}

    第一个等式左边:
    \begin{align}
        \notag a^{\dagger}\lvert\alpha\rangle\langle\alpha\rvert=&a^{\dagger}e^{-\abs{\alpha}^2}\sum_{n,m}\frac{\alpha^n}{\sqrt{n!}}\frac{(\alpha^*)^m}{\sqrt{m!}}\lvert n\rangle\langle m\rvert\\
        \notag=&e^{-\abs{\alpha}^2}\sum_{n,m}\frac{\alpha^n}{\sqrt{n!}}\frac{(\alpha^*)^m}{\sqrt{m!}}\sqrt{n+1}\lvert n+1\rangle\langle m\rvert\\
        =&e^{-\abs{\alpha}^2}\sum_{n=1}^{\infty}\sum_{m=0}^{\infty}\frac{\alpha^{n-1}}{\sqrt{(n-1)!}}\frac{(\alpha^*)^m}{\sqrt{m!}}\sqrt{n}\lvert n\rangle\langle m\rvert
    \end{align}
    第一个等式右边:
    \begin{align}
        \notag&(\alpha^*+\frac{\partial}{\partial\alpha})\lvert\alpha\rangle\langle\alpha\rvert\\
        \notag=&(\alpha^*+\frac{\partial}{\partial\alpha})e^{-\abs{\alpha}^2}\sum_{n,m}\frac{\alpha^n}{\sqrt{n!}}\frac{(\alpha^*)^m}{\sqrt{m!}}\lvert n\rangle\langle m\rvert\\
        \notag=&\alpha^*e^{-\abs{\alpha}^2}\sum_{n,m}\frac{\alpha^n}{\sqrt{n!}}\frac{(\alpha^*)^m}{\sqrt{m!}}\lvert n\rangle\langle m\rvert-\alpha^*e^{-\abs{\alpha}^2}\sum_{n,m}\frac{\alpha^n}{\sqrt{n!}}\frac{(\alpha^*)^m}{\sqrt{m!}}\lvert n\rangle\langle m\rvert+e^{-\abs{\alpha}^2}\sum_{n,m}\sqrt{n}\frac{\alpha^{n-1}}{\sqrt{(n-1)!}}\frac{(a^*)^m}{\sqrt{m!}}\lvert n\rangle\langle m\rvert\\
        \notag=&e^{-\abs{\alpha}^2}\sum_{n,m}\sqrt{n}\frac{\alpha^{n-1}}{\sqrt{(n-1)!}}\frac{(\alpha^*)^m}{\sqrt{m!}}\lvert n\rangle\langle m\rvert\\
        =&e^{-\abs{\alpha}^2}\sum_{n=1}^{\infty}\sum_{m=0}^{\infty}\sqrt{n}\frac{\alpha^{n-1}}{\sqrt{(n-1)!}}\frac{(\alpha^*)^m}{\sqrt{m!}}\lvert n\rangle\langle m\rvert,
    \end{align}
    故
    \begin{align}
        a^{\dagger}\lvert\alpha\rangle\langle\alpha\rvert=(\alpha^*+\frac{\partial}{\partial\alpha})\lvert\alpha\rangle\langle\alpha\rvert.
    \end{align}

    同理, 第二个等式左边:
    \begin{align}
        \notag\lvert\alpha\rangle\langle\alpha\rvert a=&e^{-\abs{\alpha}^2}\sum_{n,m}\frac{\alpha^n}{\sqrt{n!}}\frac{(\alpha^*)^m}{\sqrt{m!}}\lvert n\rangle\langle m\rvert a\\
        \notag=&e^{-\abs{\alpha}^2}\sum_{n,m}\sqrt{m+1}\frac{\alpha^n}{\sqrt{n!}}\frac{(\alpha^*)^m}{\sqrt{m!}}\lvert n\rangle\langle m+1\rvert\\
        \notag=&e^{-\abs{\alpha}^2}\sum_{n=0}^{\infty}\sum_{m=1}^{\infty}\sqrt{m}\frac{\alpha^n}{\sqrt{n!}}\frac{(\alpha^*)^{m-1}}{\sqrt{(m-1)!}}\lvert n\rangle\langle m\rvert
    \end{align}
    第二个等式右边:
    \begin{align}
        \notag&(\alpha+\frac{\partial}{\partial\alpha^*})\lvert\alpha\rangle\langle\alpha\rvert\\
        \notag=&(\alpha+\frac{\partial}{\partial\alpha^*})e^{-\abs{\alpha}^2}\sum_{n,m}\frac{\alpha^n}{\sqrt{n!}}\frac{(\alpha^*)^m}{\sqrt{m!}}\lvert n\rangle\langle m\rvert\\
        \notag=&\alpha e^{-\abs{\alpha}^2}\sum_{n,m}\frac{\alpha^n}{\sqrt{n!}}\frac{(\alpha^*)^m}{\sqrt{m!}}\lvert n\rangle\langle m\rvert-\alpha e^{-\abs{\alpha}^2}\sum_{n,m}\frac{\alpha^n}{\sqrt{n!}}\frac{(\alpha^*)^m}{\sqrt{m!}}\lvert n\rangle\langle m\rvert+e^{-\abs{\alpha}^2}\sum_{n,m}\frac{\alpha^n}{\sqrt{n!}}\sqrt{m}\frac{(\alpha^*)^{m-1}}{\sqrt{(m-1)!}}\lvert n\rangle\langle m\rvert\\
        \notag=&e^{-\abs{\alpha}^2}\sum_{n,m}\frac{\alpha^n}{\sqrt{n!}}\sqrt{m}\frac{(\alpha^*)^{m-1}}{\sqrt{(m-1)!}}\lvert n\rangle\langle m\rvert\\
        =&e^{-\abs{\alpha}^2}\sum_{n=0}^{\infty}\sum_{m=1}^{\infty}\frac{\alpha^n}{\sqrt{n!}}\sqrt{m}\frac{(\alpha^*)^{m-1}}{\sqrt{(m-1)!}}\lvert n\rangle\langle m\rvert,
    \end{align}
    故
    \begin{align}
        \lvert\alpha\rangle\langle\alpha\rvert a=(\alpha+\frac{\partial}{\partial\alpha^*})\lvert\alpha\rangle\langle\alpha\rvert.
    \end{align}
\end{pf}
\end{document}