\documentclass{assignment}
\ProjectInfos{量子光学}{PHYS6651P}{2021-2022 学年第二学期}{作业三}{截止时间: 2022 年 4 月 8 日 (周五)}{陈稼霖}[https://github.com/Chen-Jialin]{SA21038052}

\begin{document}
\begin{prob}
    对相干态 $\lvert\alpha\rangle$, 探测效率 $\eta$, 求探测效率
    \begin{align}
        P_m=\sum_nP_m^{(n)}\rho_{nn}=\sum_{n\geq m}^{\infty}\begin{pmatrix}
            n\\
            m
        \end{pmatrix}\eta^m(1-\eta)^{n-m}\rho_{nn}.
    \end{align}
\end{prob}
\begin{sol}
    对处于状态 $\lvert n\rangle$ 的系统, 测得 $m$ 个光子的概率为
    \begin{align}
        P_m^{(n)}=\begin{pmatrix}
            n\\
            m
        \end{pmatrix}\eta^m(1-\eta)^{n-m},
    \end{align}
    故对相干态 $\lvert\alpha\rangle$, 探测概率为
    \begin{align}
        P_m=\sum_nP_m^{(n)}\rho_{nn}=\sum_{n\geq m}\begin{pmatrix}
            n\\
            m
        \end{pmatrix}\eta^m(1-\eta)^{n-m}\rho_{nn}.
    \end{align}
\end{sol}

\begin{prob}
    单模热光场 $\rho=\sum_ne^{-n\hbar\nu/kT}(1-e^{-\hbar\nu/kT})\lvert n\rangle\langle n\rvert$ 的 Q 表示.
\end{prob}
\begin{sol}
    单模热光场的表示为
    \begin{align}
        \notag Q(\alpha)=&\frac{1}{\pi}\langle\alpha\rvert\rho\lvert\alpha\rangle=\frac{1}{\pi}\langle\alpha\rvert\sum_ne^{-n\hbar\nu/kT}(1-e^{-\hbar\nu/kT})\lvert n\rangle\langle n\vert\alpha\rangle\\
        \notag=&\frac{1-\exp\left(-\frac{\hbar\nu}{kT}\right)}{\pi}\sum_n\exp\left(-\frac{n\hbar\nu}{kT}\right)\abs{\langle n\vert\alpha\rangle}^2\\
        \notag=&\frac{1-\exp\left(-\frac{\hbar\nu}{kT}\right)}{\pi}\sum_n\exp\left(-\frac{n\hbar\nu}{kT}\right)\abs{\exp\left(-\frac{\abs{\alpha}^2}{2}\right)\frac{\alpha^n}{\sqrt{n!}}}^2\\
        \notag=&\frac{1-\exp\left(-\frac{\hbar\nu}{kT}\right)}{\pi}\exp(-\abs{\alpha}^2)\sum_n\frac{\left[\abs{\alpha}^2\exp\left(-\frac{\hbar\nu}{kT}\right)\right]^n}{n!}\\
        \notag=&\frac{1-\exp\left(-\frac{\hbar\nu}{kT}\right)}{\pi}\exp\left[-\abs{\alpha}^2\left(1-e^{-\hbar\nu/kT}\right)\right]\\
        =&\frac{1}{\pi}\frac{1}{\bar{n}+1}\exp\left(-\frac{\abs{\alpha}^2}{\bar{n}+1}\right),
    \end{align}
    其中单模热光场的平均光子数 $\bar{n}=\frac{1}{e^{\hbar\nu/kT}-1}$.
\end{sol}

\begin{prob}
    证明 Wigner 分布 $W(\alpha)$ 分布与 P 表示的关系:
    \begin{align}
        W(\alpha)=\frac{2}{\pi}\int P(\beta)e^{-2\abs{\alpha-\beta}^2}\,\mathrm{d}^2\beta.
    \end{align}
\end{prob}
\begin{pf}
    Wigner 分布是 $C_S(\beta)$ 的反 Fourier 变换:
    \begin{align}
        W(\alpha)=\frac{1}{\pi^2}\int\mathrm{d}^2\lambda\,e^{\alpha\lambda^*-\alpha^*\lambda}C_S(\lambda),
    \end{align}
    其中 $C_S(\lambda)$ 与 $C_N(\lambda)$ 之间的关系为
    \begin{align}
        C_S(\lambda)=e^{-\frac{1}{2}\abs{\lambda}^2}C_N(\lambda).
    \end{align}
    而 $C_N(\lambda)$ 的反 Fourier 变换为 $p(\beta)$:
    \begin{align}
        p(\beta)=\frac{1}{\pi^2}\int\mathrm{d}^2\lambda\,e^{\beta\lambda^*-\beta^*\lambda}C_N(\lambda),
    \end{align}
    $e^{-\frac{1}{2}\abs{\lambda}^2}$ 的反 Fourier 变换为
    \begin{align}
        \frac{1}{\pi^2}\int\mathrm{d}^2\lambda\,e^{-\frac{1}{2}\abs{\lambda}^2}e^{\beta\lambda^*-\beta^*\lambda}=\frac{2}{\pi}e^{-2\abs{\beta}^2},
    \end{align}
    故由卷积定理
    \begin{align}
        W(\alpha)=\frac{2}{\pi}\int\mathrm{d}^2\beta\,p(\beta)e^{-2\abs{\alpha-\beta}^2}.
    \end{align}
\end{pf}
\end{document}